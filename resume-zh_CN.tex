% !TEX TS-program = xelatex
% !TEX encoding = UTF-8 Unicode
% !Mode:: "TeX:UTF-8"

\documentclass{resume}
\usepackage{zh_CN-Adobefonts_external} % Simplified Chinese Support using external fonts (./fonts/zh_CN-Adobe/)
%\usepackage{zh_CN-Adobefonts_internal} % Simplified Chinese Support using system fonts
\usepackage{linespacing_fix} % disable extra space before next section
\usepackage{cite}

\begin{document}
\pagenumbering{gobble} % suppress displaying page number

\name{陈荣}

% {E-mail}{mobilephone}{homepage}
% be careful of _ in emaill address
\contactInfo{procrboo@gmail.com}{(+86) 188-1755-5359}{http://procr.cn/}
% {E-mail}{mobilephone}
% keep the last empty braces!
%\contactInfo{xxx@yuanbin.me}{(+86) 131-221-87xxx}{}
 
\section{教育背景}
\datedsubsection{\textbf{上海交通大学}, 上海}{2015 -- 2018年3月毕业}
\textit{硕士研究生}\ 并行与分布式研究所, 师从陈海波教授, GPA 2.79/3.3, rank 2 / 88
\datedsubsection{\textbf{上海交通大学}, 上海}{2011 -- 2015}
\textit{学士}\ 软件工程,GPA 3.4/4.3(84.4/100), rank 25 / 108

\section{项目经历}

\datedsubsection{\textbf{高性能虚拟机容错}}{2016年11月 -- 至今}
本项目希望解决虚拟机容错机制中备份虚拟机占用资源过多的问题,并且提升整体系统性能。

\textit{关键词} 虚拟机容错,KVM,QEMU,调度器


\datedsubsection{\textbf{Xen多核扩展性项目(华为)}}{2016年3月 -- 2016年11月}
%当前云平台的服务器CPU核数已经到达一个很高的水平(大于80核),这对内存以及CPU虚拟化的可扩展性带来了比较大的影响。
这个项目是针对于Xen超级管理者的进程间通信以及信用调度器的相关优化。
最终我们能够为之带来超过80\%的性能提升。

\textit{关键词} Xen超级管理者,进程间通信,信用调度器,锁

\datedsubsection{\textbf{VPM:非易失性内存的虚拟化}}{2015年10月 -- 2016年5月}
此工作通过修改虚拟机内核以及底层虚拟化超级管理者以支持非易失性内存的虚拟化。
%通过测试表明,在底层硬件非易失性内存有限的情况下,VPM能够为上层虚拟机带来充足的非易失性内存的抽象,并且只带来非常小的性能损耗。
这个工作的论文已经发表在系统领域高等级会议ACM Symposium on Cloud Computing 2016 (SoCC'16)上。

\textit{关键词}  非易失性内存,硬件虚拟化,半虚拟化

\datedsubsection{\textbf{宠物猫关怀系统}}{2016年11月}
微软黑客松第一名作品。旨在为工作繁忙的人士提供一套关怀宠物猫的系统。

\textit{关键词} Android,Kinect

\datedsubsection{\textbf{自在医疗}}{2016年3月 -- 2016年10月}
这个项目针对医患线上预约以及视频会诊,包含资金系统,即时通讯系统以及预约系统等。

\textit{关键词} iOS,即时通讯

\datedsubsection{\textbf{iSee}}{2013年10月 -- 2014年3月}
该应用通过地图以及增强现实来展现别人眼中的世界,通过定位和分享达到互动的目的。

\textit{关键词} Android,地缘社交


% Reference Test
%\datedsubsection{\textbf{Paper Title\cite{zaharia2012resilient}}}{May. 2015}
%An xxx optimized for xxx\cite{verma2015large}
%\begin{itemize}
%  \item main contribution
%\end{itemize}

\section{实习}
\datedline{\textit{摩根史丹利}}{2014年7月-2014年9月}
内存计算在当前业界中非常热门,实习中针对分布式内存计算框架Hazelcast进行了学习和部署,并添加了数据丢失检测的支持。

\section{论文发表}
[SoCC'16] A Case for Virtualizing Persistent Memory. Liang Liang, \textbf{Rong Chen}, Haibo Chen, Yubin Xia, Haibing Guan and Binyu Zang. 2016 ACM Symposium on Cloud Computing (ACM SoCC 2016).

\section{IT技能}
% increase linespacing [parsep=0.5ex]
\begin{itemize}[parsep=0.5ex]
  \item 编程语言: C > C++ == Java > Python
  \item 平台: Linux
  \item 开发: Vim, Git
\end{itemize}

%% Reference
%\newpage
%\bibliographystyle{IEEETran}
%\bibliography{mycite}
\end{document}
