% !TEX program = xelatex

\documentclass{resume}
%\usepackage{zh_CN-Adobefonts_external} % Simplified Chinese Support using external fonts (./fonts/zh_CN-Adobe/)
%\usepackage{zh_CN-Adobefonts_internal} % Simplified Chinese Support using system fonts

\begin{document}
\pagenumbering{gobble} % suppress displaying page number

\name{Rong Chen}

% {E-mail}{mobilephone}{homepage}
% be careful of _ in emaill address
%\contactInfo{my\_address@yuanbin.me}{(+86) 131-221-87xxx}{http://www.yuanbin.me}
\contactInfo{procrboo@gmail.com}{(+86) 188-1755-5359}{http://procr.cn/}
% {E-mail}{mobilephone}
% keep the last empty braces!
%\contactInfo{xxx@yuanbin.me}{(+86) 131-221-87xxx}{}
 
\section{Education}
\datedsubsection{\textbf{Shanghai Jiao Tong University (SJTU)}, Shanghai, China}{2015 -- Present}
\textit{M.S.} in Software Engineering, expected March 2018
\textit{GPA} 2.79/3.3, rank 2 / 88
\datedsubsection{\textbf{Shanghai Jiao Tong University (SJTU)}, Shanghai, China}{2011 -- 2015}
\textit{B.S.} in Software Engineering
\textit{GPA} 3.4/4.3(84.4/100), rank 25 / 108



\section{Experience}
\datedsubsection{\textbf{Xen Scalability Project}}{Mar. 2016 -- Present}
%\role{Summer Intern}{Manager: xxx}

As the growing number of CPUs in commodity cloud platform, the virtual machine scalability problem is becoming severe especially in memory usage and VCPU scheduling.
This project aims at optimizing the inter-processor interrupt(IPI) routine and the credit scheduler of Xen hypervisor. We get over 80\% optimum ratio in the final evaluation.

\textit{Keyword: } Xen Hypervisor, IPI, Credit Scheduler, Locking

%\begin{itemize}
%  \item Optimized xxx 5\%
%  \item xxx
%\end{itemize}

\datedsubsection{\textbf{VPM: Virtualizing Persistent Memory}}{Oct. 2015 -- May. 2016}
This work presents a study on the guest virtual machine as well as the underlying hypervisor support to virtualize persistent memory (PM). 
Performance evaluation shows that VPM can efficiently manage and multiplex PM while leading to small performance degradation even with under provisioned PM. 
The related paper "A Case for Virtualizing Persistent Memory" has been accepted by ACM Symposium on Cloud Computing 2016 (SoCC'16).

\textit{Keyword: } Persistent Memory, Virtualization, Para-virtualization



\datedsubsection{\textbf{Zizai Medical Application}}{Mar. 2015 -- Oct. 2015}
This app helps the doctors and patients make reservations and conversations online. 
The project is composed by the financial system, the instant messaging system and the reservation system.

\textit{Keyword: } iOS, IM

\datedsubsection{\textbf{iSee}}{Oct. 2013 -- Mar. 2014}
This project uses map SDK and Metaio augmented reality(AR) SDK to present and share the art. 

\textit{Keyword: } Android, Google map, Metaio AR SDK


\section{Internship}
\datedline{\textit{Morgan Stanley}}{July. 2014 -- Sep. 2014}
In memory computing is becoming more and more popular recently. During this period, I make a study of distributed system Hazelcast and add some feature for it.

\section{Publications}
[SoCC'16] A Case for Virtualizing Persistent Memory. Liang Liang, \textbf{Rong Chen}, Haibo Chen, Yubin Xia, Haibing Guan and Binyu Zang. 2016 ACM Symposium on Cloud Computing (ACM SoCC 2016).

% Reference Test
%\datedsubsection{\textbf{Paper Title\cite{zaharia2012resilient}}}{May. 2015}
%An xxx optimized for xxx\cite{verma2015large}
%\begin{itemize}
%  \item main contribution
%\end{itemize}

\section{Skills}
\begin{itemize}[parsep=0.5ex]
  \item Programming Languages: C > C++ == Java > Python
  \item Platform: Linux
  \item Development: Vim, Git
\end{itemize}

%\section{\faHeartO\ Honors and Awards}
%\datedline{\textit{\nth{1} Prize}, Award on xxx }{Jun. 2013}
%\datedline{Other awards}{2015}

%\section{\faInfo\ Miscellaneous}
%\begin{itemize}[parsep=0.5ex]
%  \item Blog: http://your.blog.me
%  \item GitHub: https://github.com/username
%  \item Languages: English - Fluent, Mandarin - Native speaker
%\end{itemize}

%% Reference
%\newpage
%\bibliographystyle{IEEETran}
%\bibliography{mycite}
\end{document}
